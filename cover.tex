

\def\date{October 2015}
\def\titulo{SleeveAR: Augmented Reality for Rehabilitation Using Realtime Feedback}
\def\avitima{Jo\~{a}o Tiago Proen\c{c}a F\'elix Vieira}

\thispagestyle{empty}

\begin{singlespace}
\vbox to\textheight{%
%--------------------------------------------------
\vskip-1.3in%---------- LOGO E NOME IST/UTL -------
%--------------------------------------------------
\hskip-17mm\vbox to50mm{
\vfil
\begin{tabular}{l}
\includegraphics[width=9cm]{imgs/Logo_IST_web.pdf}
\end{tabular}
\vfil
\vfil
}%
%--------------------------------------------------
\vskip6mm%---------- TTULO -----------------------
%--------------------------------------------------
\vbox to25mm{\LARGE\bf
\vfil
\begin{center}
\titulo
\end{center}
\vfil
}%
%--------------------------------------------------
\vskip10mm%---------- NOME E GRAU ACTUAL -----------
%--------------------------------------------------
\vbox to25mm{\large
\vfil
\begin{center}
{\Large\bf \avitima}\\   % author's name
\end{center}
\vfil
}%
%--------------------------------------------------
\vskip10mm%---------- GRAU A OBTER -----------------

%--------------------------------------------------
\vbox to8mm{\large
\vfil
\centerline{Dissertation for the Degree of Master of}
%\vskip6mm
\centerline{Engineering Systems and Computer Engineering}
\vfil
}%
% %--------------------------------------------------
\vskip10mm %---------- JURI -------------------------
% %--------------------------------------------------
\vbox to7mm{\bf
\vfil
\begin{center}
{\bf Jury}\\
\end{center}
\vfil
}%

\vbox to28mm{
\vfil
{\small
\begin{center}
\begin{tabular}{p{0.2\textwidth}l}
President: &   \\
Advisor: & Prof. Doutor Joaquim Armando Pires Jorge\\
Co-Advisor: & Prof. Doutor Artur Miguel do Amaral Ars\'enio\\
Member: &  \\
\end{tabular}
\end{center}
}
\vfil
}%
%--------------------------------------------------
\vskip28mm%---------- DATA -------------------------
%--------------------------------------------------
\vbox to4mm{\Large\bf
\vfil
\begin{center}
\date
\end{center}
\vfil
}%
%--------------------------------------------------
}%vbox
\end{singlespace}
\newpage

\chapter*{Acknowledgements}
%\chapter*{Acknowledgements}
\thispagestyle{empty}

I would like to thank my canary.
Ana Paula Morais Cabral fisioterapeuta


\vfill
\begin{flushright}
  \begin{minipage}{8cm}
    \begin{center}
      Lisboa, \date

      \avitima
    \end{center}
  \end{minipage}
\end{flushright}

\cleardoublepage

  %%%%%%%%%%%%%%%%%%%%%%%%%%%%%%%%%%%%%%%%%%%%%%%%%%%%%%%%%%%%%%%%%%%%%%%%%%%%%
  %
%%%%%                            DEDICATORIAS
 %%%
  %

\chapter*{}
\thispagestyle{empty}

% DEDICAR!
\vfill
\mbox{}
\vfill\Large
\begin{flushright}
  \begin{minipage}{8cm}
    \begin{center}

\todo{mauricio: esta pagina não precisa estar aqui}

For blablablablabla,\\

    \end{center}
  \end{minipage}
\end{flushright}
\normalsize\vfill

\cleardoublepage

  %%%%%%%%%%%%%%%%%%%%%%%%%%%%%%%%%%%%%%%%%%%%%%%%%%%%%%%%%%%%%%%%%%%%%%%%%%%%%
  %
%%%%%                                RESUMO
 %%%
  %

\chapter*{Resumo}
\thispagestyle{empty}
 
 \todo{traduzir abstract}
My abstract in Portuguese.

\newpage

  %%%%%%%%%%%%%%%%%%%%%%%%%%%%%%%%%%%%%%%%%%%%%%%%%%%%%%%%%%%%%%%%%%%%%%%%%%%%%
  %
%%%%%                            ABSTRACT
 %%%
  %

\chapter*{Abstract}
\thispagestyle{empty}

We present an intelligent user interface that allows people to perform rehabilitation exercises by themselves under the offline supervision of a therapist.  Many people suffer injuries that require rehabilitation every year. Rehabilitation entails considerable time overheads since it requires people to perform specified exercises under the direct supervision of a therapist. Thus it is desirable that patients continue performing exercises outside of the clinic (for instance at home, thus without direct therapist supervision), to complement in-clinic physical therapy.
However, to perform rehabilitation tasks accurately, patients need instant feedback, as otherwise provided by a physical therapist, to ensure correct execution of these unsupervised exercises. 
To address this problem, different approaches have been proposed using feedback mechanisms for aiding rehabilitation. 
Unfortunately, test subjects frequently report having trouble to completely understand the provided feedback which makes it hard to correctly execute the prescribed movements. 
Worse, injuries may occur due to incorrect performance of the prescribed exercises, which hinders recovery. This paper presents SleeveAR, a novel approach to provide new real-time, active feedback strategies, using multiple projection surfaces for providing effective visualizations.
Empirical evaluation compared to traditional video-based feedback shows the effectiveness our approach. Experimental results show that it is able to successfully guide a subject through an exercise prescribed (and demonstrated) by a physical therapist, with performance improvements between consecutive executions, a desirable goal to successful rehabilitation.


\newpage


\chapter*{Palavras Chave \\ Keywords}
\thispagestyle{empty}

\section*{Palavras Chave}
{\large % EM PORTUGUES

\noindent xxx

\noindent xxx

\noindent xxx

\noindent xxx

\noindent xxx

}

\section*{Keywords}

{\large % EM INGLES

\noindent xxx

\noindent xxx

\noindent xxx

\noindent xxx

\noindent xxx

}

\vfill


\cleardoublepage

\pagestyle{plain}
\pagenumbering{roman}

 
\def\contentsname{Contents}
\tableofcontents
\newpage

\listoffigures
\newpage

\listoftables

\cleardoublepage

