

\def\date{October 2015}
\def\titulo{SleeveAR: Augmented Reality for Rehabilitation Using Realtime Feedback}
\def\avitima{Jo\~{a}o Tiago Proen\c{c}a F\'elix Vieira}

\thispagestyle{empty}

\begin{singlespace}
\vbox to\textheight{%
%--------------------------------------------------
\vskip-1.3in%---------- LOGO E NOME IST/UTL -------
%--------------------------------------------------
\hskip-17mm\vbox to50mm{
\vfil
\begin{tabular}{l}
\includegraphics[width=9cm]{imgs/Logo_IST_web.pdf}
\end{tabular}
\vfil
\vfil
}%
%--------------------------------------------------
\vskip6mm%---------- TTULO -----------------------
%--------------------------------------------------
\vbox to25mm{\LARGE\bf
\vfil
\begin{center}
\titulo
\end{center}
\vfil
}%
%--------------------------------------------------
\vskip10mm%---------- NOME E GRAU ACTUAL -----------
%--------------------------------------------------
\vbox to25mm{\large
\vfil
\begin{center}
{\Large\bf \avitima}\\   % author's name
\end{center}
\vfil
}%
%--------------------------------------------------
\vskip10mm%---------- GRAU A OBTER -----------------

%--------------------------------------------------
\vbox to8mm{\large
\vfil
\centerline{Dissertation for the Degree of Master of}
%\vskip6mm
\centerline{Engineering Systems and Computer Engineering}
\vfil
}%
% %--------------------------------------------------
\vskip10mm %---------- JURI -------------------------
% %--------------------------------------------------
\vbox to7mm{\bf
\vfil
\begin{center}
{\bf Jury}\\
\end{center}
\vfil
}%

\vbox to28mm{
\vfil
{\small
\begin{center}
\begin{tabular}{p{0.2\textwidth}l}
President: &   \\
Advisor: & Prof. Doutor Joaquim Armando Pires Jorge\\
Co-Advisor: & Prof. Doutor Artur Miguel do Amaral Ars\'enio\\
Member: &  \\
\end{tabular}
\end{center}
}
\vfil
}%
%--------------------------------------------------
\vskip28mm%---------- DATA -------------------------
%--------------------------------------------------
\vbox to4mm{\Large\bf
\vfil
\begin{center}
\date
\end{center}
\vfil
}%
%--------------------------------------------------
}%vbox
\end{singlespace}
\newpage

\chapter*{Acknowledgements}
%\chapter*{Acknowledgements}
\thispagestyle{empty}

I would first like to thank Professor Joaquim Jorge and Professor Artur Ars\'enio for their guidance during this last year of work.
Secondly, I want to thank Maur\'icio Sousa for his patience and amazing guidance during the development of this work, and especially for helping me with the many technical issues found during this last year.
I must also thank my family for supporting me during this difficult year and providing me with an opportunity to attend T\'ecnico Lisboa.

I would also want to show my gratitude to Physical Therapist Ana Paula Morais Cabral for disposing of her free time to evaluate our prototype and giving such helpful feedback.

Finally, I have to thank all my friends for always being by my side during the hard, but amazing, time spent at this Institute.


\vfill
\begin{flushright}
  \begin{minipage}{8cm}
    \begin{center}
      Lisboa, \date

      \avitima
    \end{center}
  \end{minipage}
\end{flushright}

\cleardoublepage



  %%%%%%%%%%%%%%%%%%%%%%%%%%%%%%%%%%%%%%%%%%%%%%%%%%%%%%%%%%%%%%%%%%%%%%%%%%%%%
  %
%%%%%                                RESUMO
 %%%
  %

\chapter*{Resumo}
\thispagestyle{empty}

 
Todos os anos, imensas pessoas sofrem les\~oes que requerem um processo de reabilita\c{c}\~ao para recuperar totalmente.  
Esta reabilita\c{c}\~ao exige imenso tempo do paciente e fisioterapeuta, visto ser necess\'ario a constante supervis\~ao do mesmo. 
Seria vantajoso possibilitar aos pacientes a continua\c{c}\~ao do seu processo de reabilita\c{c}\~ao mesmo quando n\~ao s\~ao supervisionados por um profissional (por exemplo em casa). 
No entanto, para executar as tarefas sem supervis\~ao, os pacientes necessitam de receber feedback, algo que normalmente seria dado por um fisioterapeuta, para garantir a execu\c{c}\~ao correta dos mesmos.
Para combater este problema, v\'arias abordagens foram propostas usando mecanismos de feedback para ajudar na reabilita\c{c}\~ao de pacientes. Infelizmente, testes levados com sujeitos demonstraram alguma dificuldade em compreender totalmente o feedback fornecido, algo que torna dif\'icil a execu\c{c}\~ao de movimentos prescritos ao paciente. Al\'em disso, executar movimentos de forma incorreta num processo de reabilita\c{c}\~ao pode levar a um agravamento da les\~ao do paciente. Este trabalho introduz o SleeveAR,  uma nova abordagem capaz de fornecer feedback em tempo real usando m\'ultipla superf\'icies de proje\c{c}\~ao de forma a criar visualiza\c{c}\~ao eficazes no processo de supervis\~ao e corre\c{c}\~ao de pacientes.
A avalia\c{c}\~ao emp\'irica feita em compara\c{c}\~ao com instru\c{c}\~oes em forma de v\'ideo mostra a efic\'acia da nossa abordagem atrav\'es de resultados experimentais, foi demonstrado com sucesso que \'e poss\'ivel guiar pacientes atrav\'es de exerc\'icios previamente capturados por demonstra\c{c}\~ao de um fisioterapeuta. Al\'em disso, foram detetadas melhorias no desempenho dos exerc\'icios entre cada repeti\c{c}\~ao dos mesmos, algo bastante desejado para uma reabilita\c{c}\~ao positiva.


\newpage

  %%%%%%%%%%%%%%%%%%%%%%%%%%%%%%%%%%%%%%%%%%%%%%%%%%%%%%%%%%%%%%%%%%%%%%%%%%%%%
  %
%%%%%                            ABSTRACT
 %%%
  %

\chapter*{Abstract}
\thispagestyle{empty}

We present an intelligent user interface that allows people to perform rehabilitation exercises by themselves under the offline supervision of a therapist.  Many people suffer injuries that require rehabilitation every year. Rehabilitation entails considerable time overheads since it requires people to perform specified exercises under the direct supervision of a therapist. Thus it is desirable that patients continue performing exercises outside of the clinic (for instance at home, thus without direct therapist supervision), to complement in-clinic physical therapy.
However, to perform rehabilitation tasks accurately, patients need instant feedback, as otherwise provided by a physical therapist, to ensure correct execution of these unsupervised exercises. 
To address this problem, different approaches have been proposed using feedback mechanisms for aiding rehabilitation. 
Unfortunately, test subjects frequently report having trouble to completely understand the provided feedback which makes it hard to correctly execute the prescribed movements. 
Worse, injuries may occur due to incorrect performance of the prescribed exercises, which hinders recovery. This dissertation presents SleeveAR, a novel approach to provide new real-time, active feedback strategies, using multiple projection surfaces for providing effective visualizations.
Empirical evaluation compared to traditional video-based feedback shows the effectiveness our approach. Experimental results show that it is able to successfully guide a subject through an exercise prescribed (and demonstrated) by a physical therapist, with performance improvements between consecutive executions, a desirable goal to successful rehabilitation.


\newpage


\chapter*{Palavras Chave \\ Keywords}
\thispagestyle{empty}

\section*{Palavras Chave}
{\large % EM PORTUGUES

\noindent Reabilita\c{c}\~ao

\noindent Realidade Aumentada

\noindent Sistemas de Projec\c{c}ao

\noindent Feedback


}

\section*{Keywords}

{\large % EM INGLES

\noindent Rehabilitation

\noindent Augmented Reality

\noindent Projection-based Systems

\noindent Feedback

}

\vfill


\cleardoublepage

\pagestyle{plain}
\pagenumbering{roman}

 
\def\contentsname{Contents}
\tableofcontents
\newpage

\listoffigures
\newpage

\listoftables

\cleardoublepage

