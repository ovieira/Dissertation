%%%%%%%%%%%%%%%%%%%%%%%%%%%%%%%%%%%%%%%%%%%%%%%%%%%%%%%%%%%%%%%%%%%%%%%%%%%%%%%%%%%%%%%%%%%
%                               Conclusions - 2pp
%%%%%%%%%%%%%%%%%%%%%%%%%%%%%%%%%%%%%%%%%%%%%%%%%%%%%%%%%%%%%%%%%%%%%%%%%%%%%%%%%%%%%%%%%%%
\chapter{Conclusions}
\label{sec:conclusions}

\section*{Summary}

\section{Conclusions}
Using augmented reality with multimodal feedback for rehabilitation allows a 
patient to have a source of guidance and correction when executing 
exercises outside of a clinic. This would be preferred, as opposed to exercising with no feedback where there is no way of correcting the execution.

\todo{THE TEXT BELOW IS RELATED TO MULTIMODAL - for future work. CHANGE to discuss the innovation brought by the solution, its advantages and disadvantages}

The state of the art presents several solutions to provide guidance during movement's execution, some already applying multimodal feedback. 
Even so, several problems with the patient's perception of the feedback have been reported, and clearly there is still room for improvement when combining sources of feedback.

With our proposal, we will be able to evaluate which feedback 
combinations could be more suitable for guiding a patient while solving some of the perception problems and also 
contribute with different feedback techniques in addition to the ones observed in the state of the art.

\section{Future Work}

As future work we would like bla bla.

\todo{improvements as reffered by therapist, testing on a real therapeutic environment with real patients, extending to account for multimodal feedback, etc}