%%%%%%%%%%%%%%%%%%%%%%%%%%%%%%%%%%%%%%%%%%%%%%%%%%%%%%%%%%%%%%%%%%%%%%%%%%%%%%%%%%%%%%%%%%%
%                               Conclusions - 2pp
%%%%%%%%%%%%%%%%%%%%%%%%%%%%%%%%%%%%%%%%%%%%%%%%%%%%%%%%%%%%%%%%%%%%%%%%%%%%%%%%%%%%%%%%%%%
\chapter{Conclusions and Future Work}
\label{sec:conclusions}


Augmented reality with visual feedback for rehabilitation is expected to provide a patient with improved sources and correction when executing exercises outside of a clinic. 
This would be preferred, as opposed to exercising with no feedback where there is no way of correcting the execution.
The state of the art presents several solutions to provide guidance during movement's execution.
However, there is still room for improvement, and much research is needed to determine the optimal combination of different feedback sources
Projecting light on top of the limbs to guide a subject through a movement had some promising results, still it is difficult for patients to accurately replicate the rehabilitation exercise prescribed.

We have introduced SleeveAR, which brings augmented reality feedback and movement guidance to therapeutic and rehabilitation exercises. Not only to precisely guide people in how to perform, but also, to provide simple and clear awareness of the exactitude or the incorrectness of the required actions, using visual and audio cues.
With SleeveAR, patients are able to to formally assess feedback combinations suitable for movement guidance while solving some of the perception problems and also contribute with different feedback techniques in addition to the ones observed in the state of the art.
Furthermore, results from user tests suggests that people can replicate previously pre-recorded movements by following our proposed feedback approaches.


\todo{improvements as reffered by therapist, testing on a real therapeutic environment with real patients, extending to account for multimodal feedback, etc}

Despite that, we consider that it is both possible and interesting, as future work, to add multitude of projected surfaces (walls, furniture, or even the ceiling)  to determine their impact on the people performance and awareness during a rehabilitation session.