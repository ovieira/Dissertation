%%%%%%%%%%%%%%%%%%%%%%%%%%%%%%%%%%%%%%%%%%%%%%%%%%%%%%%%%%%%%%%%%%%%%%%%%%%%%%%%%%%%%%%%%%%
%                               Evaluation - 16pp
%%%%%%%%%%%%%%%%%%%%%%%%%%%%%%%%%%%%%%%%%%%%%%%%%%%%%%%%%%%%%%%%%%%%%%%%%%%%%%%%%%%%%%%%%%%
\chapter{Evaluation}
\label{sec:evaluation}

\section*{Summary}
%To evaluate our approach we plan to run several experimental tests within a group of, at least, ten subjects. 
%With these experiments we want to obtain statistical measures about our approach and be able to conclude 
%on which combinations of feedback could have better performance in guiding a patient.

%We will be using two different experiments in order to achieve comparable results. 
%The first experiment will consist of a \ac{PT} and a test subject. 
%The \ac{PT} will demonstrate a given exercise to the test subject and then evaluate their execution without giving any kind of feedback. 

%The second experiment will involve guiding a test subject through the same exercise as the first experiment, 
%using different combinations of visual, audio and haptic feedback.
%The resulting performance will be analyzed and several data gathered. 
%We will take into account the following metrics, (a) the trajectory error between the Exercise Model and 
%the patient's actual execution, (b) the time it takes for the patient to finish the exercise 
%and (c) the time it takes for the patient to recover to the correct position if mistaken.

%We will start by using a unimodal approach, to obtain measurements using only one of the 
%available feedback modes each time. After the unimodal experiments, we will begin the multimodal 
%approach experiments by combining the available feedback and repeat the same measurements previously done.

%Even though our main focus is to provide concurrent feedback (during the execution), its frequency will be altered in order to evaluate the patient response. Therefore, further in the end, we will provide only terminal feedback (without guiding cues during the execution) to analyze if the patient successfully learned the movement.

For the evaluation of the SleeveAR, we intended to observe how well could a subject recreate simple arm movements by only following the feedback at his disposal. In this section we will present a detailed description of the type of tests that were made, what type of information was being gathered and also highlight some of the most important critics received by a professional physical therapist after using our system.

Since the test would involve executing simple arm movements, five different exercises were created for this evaluation. \todo{colocar links de youtube?} These exercises were recorded both by video and by the SleeveAR's Learning Architecture at the same time. This way, we knew for certain that it was the same movement being stored in video and in our system.

In this chapter we present the methodology used for testing our prototype with test subjects. All the results will be discussed in order to achieve a better understanding about our prototype success. In the end a meeting with a physical therapist, from which resulted a great discussion and exchange of ideas, will also be fully reported.

\section{Methodology} \label{evaluation-methodology}

\begin{table}
\centering
\begin{tabular}{lll}
\hline
\multicolumn{1}{|l|}{\#}& \multicolumn{1}{l|}{Stage}         & \multicolumn{1}{l|}{Time}       \\ \hline
\multicolumn{1}{|l|}{1} & \multicolumn{1}{l|}{Introduction}  & \multicolumn{1}{l|}{2 minutes}  \\ \hline
\multicolumn{1}{|l|}{2} & \multicolumn{1}{l|}{SleeveAr}      & \multicolumn{1}{l|}{15 minutes} \\ \hline
\multicolumn{1}{|l|}{3} & \multicolumn{1}{l|}{Video}         & \multicolumn{1}{l|}{10 minutes} \\ \hline
\multicolumn{1}{|l|}{4} & \multicolumn{1}{l|}{Questionnaire} & \multicolumn{1}{l|}{3 minutes}  \\ \hline
\end{tabular}
\caption{SleeveAR evaluation stages}
\label{table:teststages}
\end{table}

%\begin{itemize}
%\item divided by 3 main parts
%\item executing movements following video instructions
%\item executing movements following SleeveAR
%\item answering a small form at the end
%\end{itemize}

In this section we describe what methodologies were used to test our prototype. Each of our participants followed this methods similarly.

The average time spent with each participant was approximately thirty minutes. The test was composed of four stages as we can observe in Table \ref{table:teststages}.

\begin{enumerate}
\item \textbf{Introduction}

Before the actual test, a brief explanation was given about the main goal of our thesis. The participants were also made aware of what would the full experimental test consist of.

\item \textbf{SleeveAR}

The participant would have to execute exercises, described in section \ref{evaluation-tasks}, while following our prototype real-time feedback.

\item \textbf{Video}

For each of the five exercises selected for this evaluation, the participant would have to watch a video of its execution at least two times. Then, while following the video playing, the participant would execute the same movement based on what he was observing.

\item \textbf{Questionnaire}

Finally, a small questionnaire would be filled by the participant. This questionnaire included questions about stage 2 and 3 while also providing us some information about the user's profile.

\end{enumerate}

In order to gather data for further result analysis, each execution of an exercise generated a Log with all the necessary information about the participant's movement. \todo{descrever Logs mais detalhadamente}

Even though we are ordering the stages this way, half of the participantes started by doing the third stage before the second, for the purpose of obtaining a more balanced sample of results.

\section{Performed Tasks} \label{evaluation-tasks}
For both the second and third stage described in the previous section, the same five exercises were executed.

Each exercise was simultaneously recorded witg a video camera and with motion tracking devices. Under these circunstances, we made sure that the content being stored in video format directly represented the data being stored on SleeveAR's architecture.
The tasks performed in stage 2 and 3 had the same goal of recreating the five exercises recorded.

\todo{explicar como era cada exercicio?}

\section{Environment and Participants} 



\section{Results}

\section{Validation with Physical Therapist}

We had the opportunity of meeting with a professional physical therapist which accepted our invitation to test the SleeveAR prototype and provide us with some feedback in a small interview.

Prior to the interview, it as decided that the therapist would be part of the same tests being done for our evaluation, which have already been described in this chapter \todo{ref}. Under those circumstances we were able to demonstrate the full potential of our prototype and, as a result of these tests, we received some very interesting feedback.

First of all, this prototype main vision was to prove we were able to guide subjects through pre-recorded exercises in order for them to be as close as possible of the original exercise. With this in mind, we wanted to know in what way would this kind of tool be useful in a regular physical therapy work environment. We also wanted to understand what would be missing to make SleeveAR a more complete tool for a common use along this field of rehabilitation.

We will now present the most notable feedback received, both positive and negative.




\begin{itemize}
\item \textbf{Missing feedback from one of the three axis}

For SleeveAR feedback to be fully complete, it would need to take into account the missing axis of movement in its real-time feedback. Since this prototype focused on guiding the arm through relatively simple movements, we did not detect this problem. But, consequently, in the evaluation tests, we realized that it might have helped to take this into account. In Fig \todo{fazer figura} we can see an example where, without verifying the upper arm's rotation, our system considers both arm poses to be the same. This happens because both the upper arm direction and angle between the upper and fore arm remain the same.


\item \textbf{Arm obstructs visibility}

Ocasionally, the right arm might obstruct the user's vision, making it difficult to observe the feedback being projected onto the floor. This issue could be solved by projecting all the visual feedback further away from the subject

\item there should be more fixed tracking points on the subject. mostly at the shoulder area

\todo{passar do caderno}

\item some physical therapists follow a group of standart arm movements to initially evaluate a patient's condition. with this tool, they could receive full reports with necessary data that otherwise they would have to measure physicaly

\todo{passar do caderno}
\item while working in a physical therapy gymnasium, therapists often have to look after several patients at the same time. Tools like sleeveAR could help the therapist by lowering the amout of times they have to correct a patient and, therefore, focus on another patient that might need more priority help

\todo{passar do caderno}
\item the \ac{KP} and \ac{KR} demonstrated in sleeveAr is very satisfatory and could really help in motivating a patient while showing his evolution as he keeps repeating the exercises

\todo{passar do caderno}
\end{itemize}

\section{Discussion}

results x therapist 