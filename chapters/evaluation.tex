%%%%%%%%%%%%%%%%%%%%%%%%%%%%%%%%%%%%%%%%%%%%%%%%%%%%%%%%%%%%%%%%%%%%%%%%%%%%%%%%%%%%%%%%%%%
%                               Evaluation - 16pp
%%%%%%%%%%%%%%%%%%%%%%%%%%%%%%%%%%%%%%%%%%%%%%%%%%%%%%%%%%%%%%%%%%%%%%%%%%%%%%%%%%%%%%%%%%%
\chapter{Evaluation}
\label{sec:evaluation}

\section*{Summary}
To evaluate our approach we plan to run several experimental tests within a group of, at least, ten subjects. 
With these experiments we want to obtain statistical measures about our approach and be able to conclude 
on which combinations of feedback could have better performance in guiding a patient.

We will be using two different experiments in order to achieve comparable results. 
The first experiment will consist of a \ac{PT} and a test subject. 
The \ac{PT} will demonstrate a given exercise to the test subject and then evaluate their execution without giving any kind of feedback. 

The second experiment will involve guiding a test subject through the same exercise as the first experiment, 
using different combinations of visual, audio and haptic feedback.
The resulting performance will be analyzed and several data gathered. 
We will take into account the following metrics, (a) the trajectory error between the Exercise Model and 
the patient's actual execution, (b) the time it takes for the patient to finish the exercise 
and (c) the time it takes for the patient to recover to the correct position if mistaken.

We will start by using a unimodal approach, to obtain measurements using only one of the 
available feedback modes each time. After the unimodal experiments, we will begin the multimodal 
approach experiments by combining the available feedback and repeat the same measurements previously done.

Even though our main focus is to provide concurrent feedback (during the execution), its frequency will be altered in order to evaluate the patient response. Therefore, further in the end, we will provide only terminal feedback (without guiding cues during the execution) to analyze if the patient successfully learned the movement.

\section{Methodology}

\section{Test Environment and Subjects}

\section{Results Analysis}

\section{Validation with Physical Therapist}