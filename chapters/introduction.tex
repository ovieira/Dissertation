%%%%%%%%%%%%%%%%%%%%%%%%%%%%%%%%%%%%%%%%%%%%%%%%%%%%%%%%%%%%%%%%%%%%%%%%%%%%%%%%%%%%%%%%%%%
%                               Introduction - 3pp
%%%%%%%%%%%%%%%%%%%%%%%%%%%%%%%%%%%%%%%%%%%%%%%%%%%%%%%%%%%%%%%%%%%%%%%%%%%%%%%%%%%%%%%%%%%
\chapter{Introduction}
\label{sec:intro}

\section*{Summary}

\section{Motivation}


Even though physical therapy holds a great part of a injured person's rehabilitation, 
it also requires effort from the patient to achieve a full recovery.
In fact, the patient holds great responsibility in each therapy session.
He must be ready to learn about his condition and what types of therapeutic exercises 
to do and how to perform them whenever not being supervised by a therapist (e.g., whenever performing exercises at home).
To be able to exercise alone, a patient must be taught about his body and body 
movements, i.e., he must gain \emph{body awareness}.

A person with an acceptable body awareness has a better knowledge of his body and how to correctly move it when doing exercises or other tasks that involve physical movement.
Therefore, a person is able to improve the overall quality of a given movement and to diminish unnecessary muscle tension, 
by being able to use just the muscles required to accomplish a given task~\cite{Singh2014a}.
With relatively low body awareness, it becomes hard for a patient to perform well alone and may 
end up hurting himself. %arranjar exemplos?
Consequently, to help people with low awareness 
execute \textit{prescribed} tasks, it is necessary for them to receive real-time feedback.
This feedback is usually given by a professional, 
but without their presence, it would be desirable for people to receive similar feedback from other sources to maintain a certain quality in the task execution.

\ac{AR} is a technique used to impose digital content on top of the physical world, giving the user a different perception on the subject in which \ac{AR} is being applied. 
This can manipulate the meaning or increase the amount of information available of the subject being augmented.

\ac{AR} could be a possible solution to overpass the lack of clear feedback sources when no \ac{PT} is present.
It holds great potential in the field of rehabilitation %(and several other serious matters) 
and there are already a variety of tools available to help with the development process of Augmented
Reality applications that interact with the body~\cite{Gama2012a}.

If combined with a carefully designed form of feedback for the patient,
\ac{AR} can be of great use in the rehabilitation of a person ~\cite{Sigrist2013}. 
The whole idea of it is to give more information to a person in a way that it can make the assigned task easier to do.
Therefore, the type of feedback given by the \ac{AR} system can have a great influence 
on the outcome of the task being done~\cite{Causo2011}.

As above mentioned, the feedback given to a patient helps him to correct mistakes on the performed movements.
Usually, this feedback is given by a therapist while enduring physical therapy, which can be of a visual form (the therapist demonstrating what to do), auditory form (the therapist giving orders to the patient) or physical (the therapist applying physical force to the patient). 
When performing exercises alone, a different approach must be followed on the types of feedback used, making sure that the goals 
are still achieved and the patient performs the exercises correctly.

There are multiple types of feedback, just as there are forms of expression, which means that 
there could be different combinations of feedback that could make a patient understand the task better.
Most of the traditional feedback systems, used for rehabilitation, only use a singular type of feedback,
visual or auditory~\cite{Design2005}, being these kind of systems known as \emph{unimodal feedback systems}.

By using multiple types of feedback, we can take advantage of more than one sense on the patient, 
making it possible to give more information without overloading just one sense (e.g., just using visual
cues on a screen can become overwhelming for a patient). A system like the one described is called a 
\emph{multimodal feedback system}, which, by definition, uses various sensory inputs and outputs to achieve the desired task.

\section{Research Statement}

In this work, we introduce SleeveAR, a novel approach that provides awareness feedback to aid and guide the patient during rehabilitation exercises.
SleeveAR aims on providing the means for patients to precisely replicate the exercises, especially prescribed for them by a health professional. 
Since, the rehabilitation process relies on repetition of the exercises during the physiotherapy sessions, our approach contributes to the correct performance of the therapeutic exercises while offering reports on the patient's progress. 
Also, without rendering the role of the therapist obsolete, our approach builds on the notion that with proper guidance, the patients can execute rehabilitation exercises for themselves without full time supervision. 

With this dissertation, we intend to validate the assumption that using interactive applications relying on augmented reality and real-time feedback can become a better alternative to guide patients though rehabilitation without supervision, as oppose to other sources such as video observation. We can then highlight the research statement of this dissertation as follow:


\begin{quotation}
%\noindent \textit {\textbf {Is it possible to provide a level of awareness of remote people to the point that they are able to use proximity relationships in the development of collaborative tasks?}}
\noindent \textit {\textbf { SleeveAR can help patients exercise upper limb movements with greater efficiency to that of a unsupervised rehabilitation.}}

\end{quotation}


\section{Contributions}

With the development of our SleeveAR prototype, our work provides the following contributions:

\begin{itemize}
\item \textbf{Solution for unsupervised upper-limb rehabilitation}\\
The prototype developed in our work can help patients replicate rehabilitation exercises even if they did not observed the exercise prior to their execution.

\item \textbf{Content projection on moving surfaces}\\
We present a novel technique for projecting content on top of tracked objects. With this technique, we are able to provide visual feedback on the actual upper-limb being rehabilitated.

\item \textbf{New visual feedback designs}\\
We created a group of minimalist visual cues to guide patients which cover the majority of possible arm movements.

\end{itemize}
 

\section{Publications}


The work developed in this dissertation led to a publication evaluated by an international panel of experts and accepted in a scientific conference. The publication is listed below.

\begin{enumerate}
\item \textit{Augmented Reality for Rehabilitation Using Multimodal Feedback}, \textbf{Jo\~{a}o Vieira}, Maur\'icio Sousa, Artur Ars\'enio and Joaquim Jorge, 3rd Workshop on ICTs for improving Patients Rehabilitation Research Techniques (REHAB 2015), October 2015.
\end{enumerate}


\section{Dissertation Outline}

The remaining content of this dissertation are organised as follows. 
In Chapter~\ref{section-related} we discuss related work that had influence on our approach, several state of the art works are presented and a comparison between them can be found. 
Chapter~\ref{sec:sleevear} introduces our proposed solution, SleeveAR. An approach on guiding patients through pre-recorded exercises with real-time correction feedback.
Next, in Chapter~\ref{sec:implementation}, we present our SleeveAR implementation, describing all the technology and development that allowed us to achieve our solution.
Chapter~\ref{sec:evaluation} reports the user tests conducted to evaluate our solution. 
And finally, in Chapter~\ref{sec:conclusions}, we present our conclusions and discuss our future work with SleeveAr.