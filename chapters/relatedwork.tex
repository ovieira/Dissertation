%%%%%%%%%%%%%%%%%%%%%%%%%%%%%%%%%%%%%%%%%%%%%%%%%%%%%%%%%%%%%%%%%%%%%%%%%%%%%%%%%%%%%%%%%%%
%                               Related Work - 24-32pp
%%%%%%%%%%%%%%%%%%%%%%%%%%%%%%%%%%%%%%%%%%%%%%%%%%%%%%%%%%%%%%%%%%%%%%%%%%%%%%%%%%%%%%%%%%%
\chapter{Related Work}
\label{section-related}

Motor rehabilitation, or motor re-learning, is an extensive and demanding process for a patient.
For a successful recovery, the patient must be disciplined and understand that this is a tough and painful task in which it
will normally be required to move the injured area which might cause immense pain~\cite{Singh2014a}.
Depending on the injury, recovery requires several physical therapy sessions and, after finishing them, 
the rehabilitation might have to continue at the patient's own home~\cite{Lohse2013}.

Home rehabilitation is common among injured individuals, since
attending sessions at a professional clinic is usually not enough for a full recovery.
The patient will need to add more effort outside of the clinic and continue 
exercising to avoid suffering a setback on his rehabilitation~\cite{Borghese2013} 
or to increase his recovery speed. 
Hence, the patient needs to learn what exercises to do, and how to do 
them correctly to prevent an aggravation of the injury~\cite{Tang2014a}.

There is a significant difference between rehabilitation with a \ac{PT} and without him.
The therapist, while the patient attends physical therapy, helps him to fight his pain and recover from his injury.
His role is fundamental to plan the most appropriate set of exercises the patient must perform, and to make sure they are executed correctly. 
Since the patient does not always has the ability to execute alone the exercises, or not even move without an external help,
the therapist can intervene during the session and adapt his approach according to the patient's needs~\cite{Sigrist2013}.
However, whenever the rehabilitation exercises are done at home, without the therapist presence, the patient might 
perform incorrect movements to avoid pain~\cite{Tang2014a} or might not even be able to move at all.

Repeating specific movements is a key factor in motor re-learning ~\cite{Schonauer2011a} and it 
should always be a part of the rehabilitation, whether at a clinic or at home.
However, this is also one of the main causes of deteriorated rehabilitation at home. In this case, patients tend to get bored and lose focus, 
due to both this repetition and the lack of a therapist presence to guide and motivate him~\cite{Singh2014a,Balaam2011a,Rego2010}. 
To help with this unsupervised rehabilitation work, several solutions have appeared as 
an alternative to the classic paper or video instructions. 

Using modern technologies and counting on an increasing offer in 
affordable tracking devices (e.g. Microsoft Kinect), a large
diversity of applications are being developed that aim to solve 
some of the difficulties in unsupervised rehabilitation ~\cite{Kitsunezaki2013,Borghese2013}. Several such works, focused on rehabilitation, will be discussed in the next section.



\section{Rehabilitation Systems}

Nowadays, we can observe a wide variety of rehabilitation systems which can help improve the recovery of a patient. 
Many of them have different rehabilitation goals and focus on specific injuries, e.g., stroke~\cite{Borghese2013,Design2005}, 
or limbs rehabilitation~\cite{Sadihov2013,Burke2009,Klein2013}.

The use of these systems can have a great influence in a patient's rehabilitation outside of a clinic. 
Not only it allows to maintain a certain quality on the execution of exercises, but also enables 
the patient to exercise in a comfortable environment, his home, which makes it easier to stimulate 
and motivate him during the whole process \cite{Borghese2013}.

As it has been said previously, a patient's rehabilitation is related to three concepts: repetition, feedback and motivation ~\cite{Schonauer2011a}. 
Hence, the development of a \ac{RS} should always be influenced by these three ideas and how to approach them. 

The repetitive nature of rehabilitation exercises can quickly become boring for a patient~\cite{Rego2010,Burke2009,Burdea2002}, 
therefore, there is a need for turning these exercises into something less tedious. 
When dealing with repetitive exercises, the main goal should be divided into several sub-goals. 
This way the patient keeps achieving incremental success through each repetition. 
Furthermore, compared to the approach where success is only achieved after finishing the whole task~\cite{Schonauer2011a}, he also increases his motivation.

For a patient to be informed about his execution, the feedback provided can be given in two different ways. During the execution
(concurrent feedback) or at the end (terminal feedback)~\cite{Sigrist2013}.
The concurrent feedback is given in real-time with the purpose of offering correction or guidance, 
it allows the patient to have \ac{KP}. On the other hand, terminal feedback only allows the patient 
to know if he succeeded after fully executing the task, giving him \ac{KR}~\cite{Design2005, Schonauer2011a}. 

Studies have shown a difficulty in obtaining a flawless formula when it comes to relating \ac{KP} and \ac{KR}. 
On one hand, \ac{KP} helps to accelerate the learning process of the exercise by correcting the patient in
real-time. On the other hand, prolonged \ac{KP} can create a dependency on the feedback, interfering with the 
learning process. Therefore, Sigrist~\cite{Sigrist2013} states that \ac{KP} should be reduced as the exercise
keeps advancing, gradually giving more emphasis to \ac{KR} in order to stimulate the autonomy of the patient.

Gama et al.~\cite{Gama2012a} developed a rehabilitation system in which the user position was tracked using a 
Microsoft Kinect. In this system, the user would see himself on the screen with overlaying targets that represented the desired position. 
If a incorrect posture was detected (shoulders not aligned or arm not fully stretched) he would be notified in real-time with visual messages. 
White arrows on the screen were also used as visual cues to guide the patient's arm to the target.
For each repetition, points were added to a score, depending on how well the user performed.


Another work~\cite{Klein2013} focused on rehabilitating stroke victims which normally end up with one of the arms extremely debilitated.
In this case, the main focus was to motivate the patient to move his injured arm. Even with a small range of motion, it is 
important for the patient to move it in order to improve the recovery. The patient would see a virtual arm overlaying his injured arm, which would simulate a normal arm movement. 
The virtual arm position was calculated based on a few control points around the patients shoulder and face. 
The results shown an enhancement of the shoulder range of motion in all the test subjects.

Also focused on stroke victims, Sadihov et al.~\cite{Sadihov2013} proposed a system which intended to aid in the development of 
rehabilitation exercises with an immersive virtual environment. In this case, using a haptic glove with vibration capabilities.  
Three virtual games were developed where the user could interact with his hand. The vibrating motors on the glove 
were activated according to what happened in the game. For example, in one of the games, the user had to hit 
the incoming meteors with his hands to protect a village and every time one meteor collided with the avatar's hand,
the haptic glove would also vibrate. This enabled patients to feel more connected with the game and thus become more motivated to exercise their debilitated limb.

Due to improving motivation and diminishing boredom while rehabilitating, using serious games
has been a trend in the latest years as we can see for the several 
research published around the theme~\cite{Schonauer2011c, Ma2008, Burke2009, Borghese2013, Lange2012}. 
%%physio@home

Tang et al.~\cite{Tang2014a} developed Physio@Home, a guidance system to help patients execute movements by following
guidelines. The patient would see himself on a mirror and, on top of the reflection, visual cues that 
indicated the direction to which the arm should move. The exercises were pre-recorded by another person and 
then replicated by the patient. If the patient started moving in the wrong direction, a red stick figure resembling the user's arm would appear in the nearest arm position where he should be.
Even though a error metric was developed to compare pre-recorded exercises with user's attempt, in nowhere was stated these metric were provided to the user. Therefore, Physio@Home only provided feedback during the performance and not after.


Most approaches usually rely on Augmented Reality technology, enhancing our perception of the real world
by adding information or manipulating our surroundings.  

\section{Augmented Reality}
\label{RW-AR}

Nowadays, Augmented Reality applications 
are being developed for several fields such as entertainment, games, military training 
and medical procedures~\cite{Guimaraes2014a, Rego2010}. 
It is rather hard to list all the possibilities of augmented reality when its limit can only be imposed by one's creativity (if we ignore technological limits).
Its use can, for example, allow a surgeon to monitor a patient's heartbeat 
and temperature in real time, or even help a military jet pilot to see targets info in his visor while flying.

In the rehabilitation field, \ac{AR} has been increasingly the target of research works.
The possibility of creating interactive and immersive environments
allowed to solve some of the difficulties of classic rehabilitation.

For example, a PT could have a better judgment over a patient's condition if he 
had access to the patient's real time data regarding body posture, joints angles or movements in general, thus helping him to better evaluate the patient's condition. 
Without augmented information, this type of information could only be obtained through naked eye 
estimates or by using regular video recordings.

A common approach in this field is to use augmented reality mirrors. 
This is inspired by the need for a patient to be able to see his body while
learning and executing movements, mainly to help with spatial awareness. We can often
see mirrors placed in physical therapy clinics for this reason and, therefore, 
augmented reality mirrors can be considered an ''evolution'' of the classic mirror.
But not only in rehabilitation can \ac{AR} mirrors be useful: we can observe 
the presence of mirrors in any activity that requires movement learning, like dancing or martial arts.

Next, we present some examples where augmented reality mirrors were used.



\subsection{Augmented Reality Mirrors}
\label{RW-mirrors}
Mirrors allow a person to have visual feedback of his body. It enhances the spatial awareness which is useful for motor learning activities.

The concept of an \ac{AR} mirror does not necessarily require an 
actual physical mirror to be implemented. 
Its functionality can be easily simulated by a virtual mirror 
which consists in capturing images with a camera and projecting 
them in real-time on a screen facing the user, 
giving him the perception of a real mirror. 

Nevertheless, there has been implementations of \ac{AR} in actual physical mirrors \cite{Anderson}. 
This was achieved by creating a mirror with a partially reflective layer facing the user and a diffuse layer in the back. 
The reflective layer maintained a mirror natural reflection while a light-projector projected images onto the diffuse layer.
The result was a mixture of the user's reflection with virtual images.

Virtual mirrors could be considered an easier alternative to implement than the one used above. 
By allowing any screen to turn into a mirror with the use of a color camera, 
it is normal that this seems to be the most common approach.

%The whole idea of using mirrors is to 
%
%They have been used for a wide variety of subjects such as anatomy education \cite{blum2012}

\ac{AR} makes it possible to add more capabilities to the classic mirror. 
In a visual feedback perspective, we can generate virtual images on top of the reflection (for instance, for guiding purposes). 
There has been already applications that make use of \ac{AR} mirrors to guide a user, whether it be for rehabilitation \cite{Tang2014a,Velloso2013,Klein2013} or for other types of interaction not focused on rehabilitation \cite{Alhamid2012a,blum2012}.

Although \ac{AR} mirrors have proven to be useful for visual feedback, there are some limitations. 
An obvious limitation of this virtual alternative is the ``reflection`` dependency on the camera direction, 
so that if a user looks at the screen from a different direction
other than directly forward, the reflection would not be correct.
The lack of depth perception means that 3-dimensional movements are more difficult to be guided by virtual images on a flat screen. 
We can conclude that \ac{AR} mirrors are more suitable for 2-dimensional movements. One possible way of solving this limitation, 
is to combine other augmented reality sources in a way that they can complement each other and not be stuck within a screen.


\subsection{Augmented Reality with Light-Projectors}
\label{RW-projectionmapping}


\begin{figure}[!t]
    \begin{center}
        \includegraphics[width=\textwidth]{imgs/lightguide}
    \end{center}
    \caption{LightGuide Visual Cues, Sodhi et. al \cite{Sodhi2012}.}
    \label{fig:lightguide}
\end{figure}


Using light-projectors for augmented reality has enabled the creation of very interesting applications. 
Through techniques of projection mapping, it became possible to turn any irregular surface into a projection screen.
We can observe this technique being applied in different objects. It is regularly used for live shows 
using buildings as the screen. One example could be the promotion of the movie "The Tourist" where 
projection mapping was applied to an entire building \cite{projectionmapping_building}. But it can 
also be used on the human body to obtain interesting effects. Barbosa \cite{projectionmapping_face} 
used projection mapping to shoot a music video in just one take where mesmerizing effects were applied 
onto the singer just by using a projector. 
By using projection mapping we can alter an object perception and create optic illusions.

This kind of technique can bring great benefits to fields that rely on guiding feedback by being able 
to focus projection on a body part for example, just as it is necessary in rehabilitation systems.
But for it to be useful, the projection mapping should be interactive and done in run-time instead of 
being pre-recorded like the examples above.

LightGuide \cite{Sodhi2012}, explored the use of projection mapping in a innovative way. 
The projection was made onto the user, using his body as a projection screen. 
Real-time visual cues were projected onto the user's hand in order to guide him through the desired movement. 
By projecting the information in the body part being moved, the user could keep a high level of concentration 
without being distracted by external factors.
As we can in the examples shown in Figure~\ref{fig:lightguide}, different types of visual cues were developed, having in mind movements that demanded degrees of freedom over 3 dimensions. 
For each dimension a different design was planned so that the user could understand clearly to what direction 
should his hand move.

To apply real-time projection mapping onto a moving body part, its 
position must be known at all time to make sure the light projector is 
illuminating the correct position. 
For this, motion tracking devices are used which enable to record the movement of, in this case, a person. 

\section{Tracking Techniques} 

Tracking devices have enabled the development of more immersive interactive applications.
Whether it be for entertainment or more serious matters, the possibility of interacting with 
an interface without using any kind of handheld devices can greatly enhance a user experience.

Nowadays it is possible to obtain affordable tracking devices such as Microsoft's Kinect, which 
can provide full skeleton tracking without the use of any kind of special equipment.
As opposed to more professional solutions that require special suits with markers, but provide a more accurate tracking. 
Even so, studies have shown that Kinect has an acceptable accuracy in comparison with other motion tracking alternatives 
and can be considered a valuable option for its low price and easy portability \cite{Scano2014,Chang2012a}.

To provide interactive content, the user's body must be detected and its position passed as input. This input normally consists of several tracking points which represent body joints. Their relative position between one another give us a representation of the user's current body posture, since each connection between two joints can be considered a bone as we can see in the 
Fig. \ref{fig:joints}.
In the Kinect's case, being a markerless tracking device, these joints are defined through software.


Aiming at rehabilitation, using tracking technology could enable applications to track a therapist's demonstration of a given movement prescribed to a patient.
Then, when the patient performed it, his movement could also be tracked and compared to the therapist's demonstration to detect possible errors.
For this to be possible, several factors have to be taken into account like the possible physical differences between both. If we were to make a ``blind`` comparison between both skeletons, the results would not be accurate.

Two comparison methods that can be used to address the aforementioned problem are described hereafter.


\subsection{Skeleton Comparison Methods}
\label{sec:skeletoncomparison}

\begin{figure}[!t]
    \begin{center}
        \includegraphics[width=0.3\textwidth]{imgs/joints}
    \end{center}
    \caption{Joints position from Kinect}
	\label{fig:joints}
\end{figure}

In order to facilitate the description of the following methods, we will consider two given skeletons named SK1 and SK2, both with the same number of defined joints and where SK2 wants to mimic SK1's pose.

The first method of measuring differences between skeletons is through the usage of their joints position. 
As we can see in Fig.\ref{fig:sk1sk2}, SK1 and SK2's arms are not in identical positions. 
If we consider the euclidean distance between joints J11 and J12, they might never be considered equal if their arms have different lengths. 
If the euclidean distance never reaches zero, these two joints might never overlap. As we can see in Fig. \ref{fig:sk1sk2diff}, 
when both skeletons achieve identical pose, there still exist a distance \textit{A} between them, therefore by using the joint position 
this would not be an identical pose between them.
To solve this problem, another method must be used for comparison, 
which relies on other measurements not dependent of, i.e. invariant to, joint specific position. 

If we use the joint angles for comparison, it is possible to achieve better results due to 
the physical differences not influencing the measurements \cite{Borghese2013}. 
In this case, looking once more at Fig. \ref{fig:sk1sk2diff}, 
if we take into account the joint angle \textit{B}, both skeletons can be considered to have an identical posture, even though they have different arms length.

The accuracy of skeleton comparison has a important role in rehabilitation systems 
where the patient will be corrected in real-time. His body tracking data will be the 
base of the system behaviour and it will influence how it responds to the patient. 
Next, we will analyse the state of art concerning several different approaches for 
the provisioning of feedback information to the patient.


\begin{figure}[!t]
    \begin{center}
        \includegraphics[width=0.6\textwidth]{imgs/SK1SK2}
    \end{center}
	\caption{SK1 shows desired pose, SK2 midway to achieving it.}
	\label{fig:sk1sk2}
\end{figure}



\begin{figure}[!t]
    \begin{center}
        \includegraphics[width=0.5\textwidth]{imgs/sk1sk2diff}
    \end{center}
	\caption{SK1 and SK2 overlapped}
	\label{fig:sk1sk2diff}
\end{figure}



\section{Information Feedback}
\label{RW-MF}

The basic goal of feedback is, as the name says, to feed information back to the user. 
It does not have to be in a textual form even though that is the most 
common form of feedback used for humans. We can receive information 
by using different means of communication. 
Everyday we are constantly processing information through a wide variety of ways like when 
we know someone is at the door because we hear the door ring bell or we recognize a friend within our sight.
Our senses are constantly at work to provide us information about our surroundings.
We can think about our senses as some sort of input sensor, each one designed for a specific type of information.

The information we receive from around us has an influence on our behaviour.
When a patient is attending physical therapy, the therapist is constantly interacting with him. 
This interaction is important in order for the patient to keep doing correctly the rehabilitation.
Not only does the therapist tells him what to do but also demonstrates it and, whenever necessary, physically corrects him.
What we observe here is the use of three different types of feedback being given to the patient - audio, visual and haptic,
each one being interpreted by hearing, sight and touch respectively.

For an automated rehabilitation system to successfully work, these interactions must 
be simulated by other sources of feedback, in a way that the patient understands 
what he must do without the presence of the therapist.

Visual feedback information is often used in rehabilitation systems to communicate with a user \cite{Design2005}. 
As one example of visual feedback on an \ac{AR} perspective, we have the overlaying of 
information on an interactive mirror for the user to analyze his performance in real-time \cite{Anderson,Tang2014a,Velloso2013,Klein2013,Alhamid2012a,blum2012}. 

Since there are multiple forms of giving feedback to a user, we can see examples where more than one are used at the same time.
Combining forms of feedback can provide better understanding of the tasks to a user by minimizing the amount of 
information given in a visual form and, instead, distribute it. 
But if not designed with caution, a system can end up overloading the user with too much information at the same time. 

\subsection{Feedback Applications}



Sigrist et al.~\cite{Sigrist2013} suggests that different types of feedback can complement each other and enhance the user comprehension. 
Alhamid et al.~\cite{Alhamid2012a} introduced an interface between a user and biofeedback sensors (sensors that are able to measure physiological functions). 
Even though it is not aimed for rehabilitation, his approach on user interaction can be analyzed.
Through this interface, the user was able to access data about his body and health thanks to the measurements made by the biofeedback sensors.
This system was prepared to interact with the user using multiple response interfaces, each one intended for specific purposes.
The visual interface relied on a projector that showed important messages and results from the biofeedback measurements.
In the other hand, the audio interface was responsible for playing different kinds of music through speakers. 
The music was selected depending on the user's current state. For example, if high levels of stress are detected,  calming music would be played to help the user relax.

One of the most common approaches on visual feedback is the augmented mirror approach already discussed. 
Its common use is justified by the fact that even without overlaying virtual images, it enables the user to have a spatial awareness of his own body.
But since a simple reflection does not provide guidance, we could observe several examples of augmented feedback being applied to the mirror.
Physio@Home, the work of Tang et al.~\cite{Tang2014a}, explored two different designs for visual guidance on a mirror aimed at upper-limbs movement.
Their first iteration consisted of virtual arrows that pointed at the targeted position for the user's hand.
The second provided a trace of tubes placed along a path which represented the complete movement to be performed by the user's arm.
In both cases it was detected some difficulty in depth perception. 
This kind of visual cues has proven not to be suitable for exercises 
where the user had to move his arm towards the camera or when he had to contract it.

Anderson et al.~\cite{Anderson} tried to provide a more detailed visual feedback 
by using a full virtual skeleton placed over the user reflection. In this case the goal 
was to mimic the skeleton's pose and hold it for a specific time.
To diminish the lack of depth perception, a second tracker was placed on the user's side. 
Every time the system detected a large error on the z-axis, a window would appear with a 
side-view of both the virtual and user's skeleton for him to be corrected.

Unlike the previous approach, LightGuide~\cite{Sodhi2012} does not rely on interactive mirrors or screens to apply its visual feedback. 
By using a depth-sensor camera and a light projector, they were able 
to project information on the user's hand. 
This approach was able to guide the hand through a defined 
movement by projecting visual cues. 
All the information projected on the hand was being updated in real-time influenced by the current position given by the tracking device.
The visual cues varied according to the desired direction of the movement. 
If the current movement only required back and forward motion, only one dimension was being used. 
Therefore, the visual cue would only inform the user where to move his 
hand in the z axis through a little arrow pointing to the correct position.
Two dimensional movements would combine the first visual cue by virtually painting the
remaining of the hand with a color pattern. The portion of the hand closer to the
desired position, would be painted with a different color than the remaining portion.
They concluded that by using LightGuide, most of the users could better execute a certain 
movement than if they were following video instructions.

\section{Related Work Overview}




After analyzing several examples of feedback approaches, it is possible to make some conclusions about their usefulness, whether it be rehabilitation-oriented or not.

Indeed, each of the three types of feedback observed, namely visual, audio and haptic, have shown to be more suitable for different purposes.
Visual feedback appears to be normally used in regard to spatial information, due to the perception of space being the most precise when using the sense of sight. 
For this reason, the best option to guide a patient through movements seems to be by using visual guidance.
But it is important to note that visual feedback still is a rather broad concept, therefore we could observe different takes on the whole subject of visual guidance.

The \ac{AR} mirror, discussed at Section~\ref{RW-mirrors}, is the most common 
solution to provide visual feedback, given that it can add information to the already present
mirror reflection. Even though a problem seems to persist throughout the several examples, namely the lack of depth perception.
But other approaches might have a chance of solving this problem if one tries to combine them both.

The use of projection mapping,might bring some improvements to visual feedback. 
Based on the \textit{LightGuide} from Sodhi et al. \cite{Sodhi2012}, there are reasons to be optimist 
about this possibility. With LightGuide, projection mapping was applied only to the hand, but their results 
are a good motivation to extend projection mapping to the full upper-limb and experiment with it.
This technique has been normally used for entertainment and, to our knowledge, has not been fully explored 
in a rehabilitation context. 

%Unlike visual, audio and haptic feedback do not require the patient to keep his focus on the external feedback source. 
%It allows him to concentrate on his own movement and receive feedback without looking at it, although it becomes 
%more difficult to guide the user through more complex movements.

Audio feedback, even though being used in several of the described works, did not have such an important role as the visual feedback. 
Despite not normally being the main source of a patient's guidance, there is significant evidence that a rehabilitation system 
can benefit from using audio for some of its needs.
Sound does not only help with the immersion in a rehabilitation environment but it is also useful to alert the patient about specific events. It 
can provide the patient with a better control of his timing when necessary, for instance to inform him 
of the right moment to evade an obstacle \cite{Wellner2007a}. This application of audio feedback is backed 
up by the fact that the sense of hearing provides a great perception of temporal information \cite{Sigrist2013}.

Our approach follows the work of Sodhi et al.~\cite{Sodhi2012} (LightGuide) and Tang et al.~\cite{Tang2014a} (Physio@Home), both of them addresses movement guidance. 
But both they lack performance review tools, feature much needed during the rehabilitation process.
Also they assume that users always execute almost perfect movements, since the error feedback relies only in pointing to the direction of the pre-recorded exercise.
In addition, the Physio@Home, mirror metaphor, provides for poor depth perception. In Table~\ref{table:rwcomparison} we compare the extracted features from our main researched works and compare it to our approach.\todo{escrever mais aqui sobre a comparacao}

\begin{table}[!t]
\centering
\scalebox{0.65}{
\begin{tabular}{l|c|c|c|c|c|}
\cline{2-6}
                                  & \textbf{Pre-Recorded Exercises} & \textbf{Movement Guidance} & \textbf{Error Feedback} & \textbf{Performance Review} & \textbf{Depth Perception} \\ \hline
\multicolumn{1}{|l|}{Physio@Home} & \ding{51}                       & \ding{51}                  & \ding{51}               &                             &                           \\ \hline
\multicolumn{1}{|l|}{LightGuide}  &                                 & \ding{51}                  &                         &                             & \ding{51}                 \\ \hline
\multicolumn{1}{|l|}{SleeveAR}    & \ding{51}                       & \ding{51}                  & \ding{51}               & \ding{51}                   & \ding{51}                 \\ \hline
\end{tabular}
}
\caption{Feature comparison with our approach}
\label{table:rwcomparison}
\end{table} 


\section{Summary}

In this Chapter, we provided an overview of the state of the art regarding our work. 
Firstly, we review the existing rehabilitation systems focused on helping patients in recovering with a less dependency on professional supervision.
Secondly, we described the state-of-the-art regarding the use of Augmented Reality in a rehabilitation context. Also, we described some interesting works that, even tought not aimed for rehabilitation, could be applied in this same context.
Thirdly, we provided some insight related to tracking techniques and possible obstacles in comparing different subjects due to physical differences.
Fourthly, we focused on different ways of providing feedback to users and describe some works that used real-time feedback to inform users about their activity.
Finally, we make a features comparison between, what we considered, the main presented works and our approach.
Following this chapter, we describe our proposed approach.