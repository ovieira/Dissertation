
%%%%%%%%%%%%%%%%%%%%%%% file typeinst.tex %%%%%%%%%%%%%%%%%%%%%%%%%
%
% This is the LaTeX source for the instructions to authors using
% the LaTeX document class 'llncs.cls' for contributions to
% the Lecture Notes in Computer Sciences series.
% http://www.springer.com/lncs       Springer Heidelberg 2006/05/04
%
% It may be used as a template for your own input - copy it
% to a new file with a new name and use it as the basis
% for your article.
%
% NB: the document class 'llncs' has its own and detailed documentation, see
% ftp://ftp.springer.de/data/pubftp/pub/tex/latex/llncs/latex2e/llncsdoc.pdf
%
%%%%%%%%%%%%%%%%%%%%%%%%%%%%%%%%%%%%%%%%%%%%%%%%%%%%%%%%%%%%%%%%%%%


\documentclass[runningheads,a4paper]{llncs}

\usepackage{amssymb}
\setcounter{tocdepth}{3}
\usepackage{graphicx}
\usepackage{epstopdf}
\usepackage[printonlyused]{acronym}

%%TODO NOTES
\usepackage[colorinlistoftodos]{todonotes}

%%



\usepackage{url}
\urldef{\mailsa}\path|felixvieira@tecnico.ulisboa.pt|    
\newcommand{\keywords}[1]{\par\addvspace\baselineskip
\noindent\keywordname\enspace\ignorespaces#1}

\begin{document}

\mainmatter  % start of an individual contribution

% first the title is needed
\title{Augmented-Reality Upper-Limb Rehabilitation with Multimodal Feedback}

% a short form should be given in case it is too long for the running head
\titlerunning{\doctitle}

% the name(s) of the author(s) follow(s) next

\author{Jo\~{a}o Vieira - 67019}
%
\authorrunning{\doctitle}
% (feature abused for this document to repeat the title also on left hand pages)

% the affiliations are given next; don't give your e-mail address
% unless you accept that it will be published
\institute{Dep. of Computer Science and Engineering, \\ Instituto Superior T\'{e}cnico, Universidade de Lisboa\\
Lisbon, Portugal\\
\mailsa\\
\url{http://tecnico.ulisboa.pt}}

%
% NB: a more complex sample for affiliations and the mapping to the
% corresponding authors can be found in the file "llncs.dem"
% (search for the string "\mainmatter" where a contribution starts).
% "llncs.dem" accompanies the document class "llncs.cls".
%

\toctitle{\doctitle}
\tocauthor{Authors' Instructions}
\maketitle

%
%\begin{abstract}
%The abstract should summarize the contents of the paper and should
%contain at least 70 and at most 150 words. It should be written using the
%\emph{abstract} environment.
%\keywords{We would like to encourage you to list your keywords within
%the abstract section} 
%\end{abstract}

\begin{abstract}
After being discharged from physical therapy, it is normally required of a patient to execute more exercises 
outside of the clinic and without a therapist's supervision. 
This extra effort may help boost his recovery but it can also damage him if done incorrectly. 
This depends on the patients body awareness.
Our goal is to develop a Augmented-Reality application to help a patient correctly execute
upper-limb movements by guiding him with multimodal feedback. 
\keywords{We would like to encourage you to list your keywords within
the abstract section}
\end{abstract}


\section{Introduction}

%porquê da tese? (relacionar FT)

Even though physical therapy holds a great part of a injured person's rehabilitation, 
it alone cannot bring a full recovery if there is no effort from the patient. 
In the therapy sessions the patient holds great responsibility of its own recovery.
He must be ready to learn about his condition and what types of therapeutic exercises 
to do and how to perform them when not being supervised by a therapist.To be able to exercise alone, a patient 
must be taught about his body and body movements, i.e., he must gain Body Awareness.

Body Awareness describes the concept of a person's knowledge about his own body,
and its right behaviour for certain movements and tasks. It helps the person improve the benefits of
movement and diminishes unnecessary muscle tension, being able to use just the necessary muscles for the given task.\todo{Cortar Frase}
%referencia a Motivating People with Chronic Pain to do Physical Activity : Opportunities for Technology Design  - section enhancing awareness
Without this, it becomes hard for a patient to perform well when alone and may end up doing more damage than good.
To help people with low awareness execute the given tasks, it's necessary for them to receive real-time feedback.
This feedback is normally given by a professional, but without its presence, other solutions had to be created.\todo{Necessário falar mais do tema ou posso abordar AR agora?}

Augmented Reality is a great solution for this problem, has been increasing over the last years.
It holds great potential in the field of rehabilitation (and several other serious matters) and 
there are already a variety of tools available to help with the development process of Augmented
Reality applications that interact with the body. %referencia a "Guidance and Movement Correction Based on Therapeutics Movements for Motor Rehabilitation Support Systems"

The goal of \ac{AR} is to impose digital content on top of the real world content,
giving the user a different perspective on the subject in which the \ac{AR} is being
applied. This can manipulate the meaning or increase the amount of information available
of what is being seen.
\todo{Aqui queria falar do que pode ser usado com \ac{AR} 
como por exemplo informação do paciente para um cirurgiao, 
monumentos destruidos q sao parcialmente contruidos digitalmente, etc.
Devo dar esses exemplos aqui ou só no related work?}
\ac{AR} can be of great use in the rehabilitation of a person, if combined 
with a acceptable form of feedback for the patient. The whole idea of it is to give extra information
to a person in a way that it can make the assigned task easier to do. \todo{Encontrar alternativa a "extra information"}
Therefore, the type of feedback given by the \ac{AR} system can have a great influence 
on the outcome of the task being done.%ref Vibrotactile feedback and visual cues for arm posture replication
\todo{A partir daqui introduzo o conceito multimodal feedback e termino a introdução?}





\section{Related Work}
\subsubsection{Augmented Reality}
\subsubsection{Multimodal Feedback}


%Nas sessões de fisioterapia, os utentes têm grande responsabilidade na sua recuperação.
%O ensino e a educação ao utente não devem ser descurados, devendo ser-lhes ensinado alguns exercícios 
%para fazer e casa, bem como estratégias para o dia-a-dia/trabalho. É importante o seu encorajamento, 
%pois a estimulação frequente e a repetição melhoram a´consciencialização e a habilidade para a activação, mais do que um exercício isolado uma vez ao dia (CATALANO & KLEINER, 1984, cit. por MAGAREY & 
%JONES, 2003; SHUMWAY-COOK& WOOLLACOTT, 2001).

%The patients are often required to also exercise at home with a given set of movements to help with the recovery.

%The main problem when doing exercises alone is the probability of doing them wrong, 
%depending on the awareness of the patient. Even with simple movements, there are 
%correct ways of doing them when enduring in a rehabilitation process. 

%qual é o papel do multimodal feedback?



%\section{LNCS Online}

%The online version of the volume will be available in LNCS Online.
%Members of institutes subscribing to the Lecture Notes in Computer
%%Science series have access to all the pdfs of all the online
%publications. Non-subscribers can only read as far as the abstracts. If
%they try to go beyond this point, they are automatically asked, whether
%they would like to order the pdf, and are given instructions as to how
%to do so.

%Please note that, if your email address is given in your paper,
%it will also be included in the meta data of the online version.

%\section{BibTeX Entries}

%The correct BibTeX entries for the Lecture Notes in Computer Science
%volumes can be found at the following Website shortly after the
%publication of the book:
%\url{http://www.informatik.uni-trier.de/~ley/db/journals/lncs.html}

%\subsubsection*{Acknowledgments.} The heading should be treated as a
%subsubsection heading and should not be assigned a number.

\section{The References Section}\label{references}

In order to permit cross referencing within LNCS-Online, and eventually
between different publishers and their online databases, LNCS will,
from now on, be standardizing the format of the references. This new
feature will increase the visibility of publications and facilitate
academic research considerably. Please base your references on the
examples below. References that don't adhere to this style will be
reformatted by Springer. You should therefore check your references
thoroughly when you receive the final pdf of your paper.
The reference section must be complete. You may not omit references.
Instructions as to where to find a fuller version of the references are
not permissible.

We only accept references written using the latin alphabet. If the title
of the book you are referring to is in Russian or Chinese, then please write
(in Russian) or (in Chinese) at the end of the transcript or translation
of the title.

The following section shows a sample reference list with entries for
journal articles \cite{jour}, an LNCS chapter \cite{lncschap}, a book
\cite{book}, proceedings without editors \cite{proceeding1} and
\cite{proceeding2}, as well as a URL \cite{url}.
Please note that proceedings published in LNCS are not cited with their
full titles, but with their acronyms!

\begin{thebibliography}{4}

\bibitem{jour} Smith, T.F., Waterman, M.S.: Identification of Common Molecular
Subsequences. J. Mol. Biol. 147, 195--197 (1981)

\bibitem{lncschap} May, P., Ehrlich, H.C., Steinke, T.: ZIB Structure Prediction Pipeline:
Composing a Complex Biological Workflow through Web Services. In: Nagel,
W.E., Walter, W.V., Lehner, W. (eds.) Euro-Par 2006. LNCS, vol. 4128,
pp. 1148--1158. Springer, Heidelberg (2006)

\bibitem{book} Foster, I., Kesselman, C.: The Grid: Blueprint for a New Computing
Infrastructure. Morgan Kaufmann, San Francisco (1999)

\bibitem{proceeding1} Czajkowski, K., Fitzgerald, S., Foster, I., Kesselman, C.: Grid
Information Services for Distributed Resource Sharing. In: 10th IEEE
International Symposium on High Performance Distributed Computing, pp.
181--184. IEEE Press, New York (2001)

\bibitem{proceeding2} Foster, I., Kesselman, C., Nick, J., Tuecke, S.: The Physiology of the
Grid: an Open Grid Services Architecture for Distributed Systems
Integration. Technical report, Global Grid Forum (2002)

\bibitem{url} National Center for Biotechnology Information, \url{http://www.ncbi.nlm.nih.gov}

\end{thebibliography}

\section*{List of Acronyms}
\begin{acronym}
\acro{AR}{Augmented Reality}
\acro{PT}{Physical Therapy}
\end{acronym}

\section*{Appendix: Springer-Author Discount}

LNCS authors are entitled to a 33.3\% discount off all Springer
publications. Before placing an order, the author should send an email, 
giving full details of his or her Springer publication,
to \url{orders-HD-individuals@springer.com} to obtain a so-called token. This token is a
number, which must be entered when placing an order via the Internet, in
order to obtain the discount.

\section{Checklist of Items to be Sent to Volume Editors}
Here is a checklist of everything the volume editor requires from you:


\begin{itemize}
\settowidth{\leftmargin}{{\Large$\square$}}\advance\leftmargin\labelsep
\itemsep8pt\relax
\renewcommand\labelitemi{{\lower1.5pt\hbox{\Large$\square$}}}

\item The final \LaTeX{} source files
\item A final PDF file
\item A copyright form, signed by one author on behalf of all of the
authors of the paper.
\item A readme giving the name and email address of the
corresponding author.
\end{itemize}
\end{document}
