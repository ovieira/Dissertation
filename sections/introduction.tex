\section{Introduction}
\label{section-introduction}

Even though physical therapy holds a great part of a rehabilitation process,
the individual effort from the patient also holds great responsibility in its own recovery.
The patient must be ready to learn about his condition and how to perform the therapeutic exercises prescribed without the need for professional supervision.

It is hard for the patient, alone and without significant body awareness, to perform correctly the exercises. Without real-time feedback, the patient may end up hurting himself.
This feedback is usually given by a professional, but without their presence, it would be desirable for people to receive similar feedback from other sources, in order to maintain a certain quality in the task execution.

Augmented Reality (AR) could be a possible solution to overpass the lack of clear feedback sources when no physical therapist is present.
It holds great potential in the field of rehabilitation. In addition, there are already a variety of tools available to help with the development process of Augmented Reality applications that interact with the body~\cite{Gama2012a}.
If combined with a carefully designed form of feedback for the patient, \ac{AR} can be of great use in the rehabilitation of a person~\cite{Sigrist2013}. 
The whole idea is to give more information to a person in a way that it can make the assigned task easier.
Usually, this feedback is given by a therapist while enduring physical therapy, which can be of a visual form (the therapist demonstrating what to do), auditory (the therapist giving orders) or physical (the therapist applying physical force). 
For unsupervised exercises, a different approach must be followed on the types of feedback used, making sure that the goals are still achieved and the patient performs correctly.
By using multimodal feedback, we can take advantage of senses, while providing more information without overloading just one sense (e.g., just using visual cues on a screen can become overwhelming for a patient). 
A system like the one described is called a \emph{multimodal feedback system}, which, by definition, uses various sensory inputs and outputs to achieve the desired task.
\subsection{Goals}
%qual é o objectivo, qual é a hip que quero provar
%obj e sub-obj

With this work we want to explore the possible benefits of using augmented reality with multimodal 
feedback for guidance in a rehabilitation context.

We plan to analyze several solutions in a variety of fields that rely on feedback for user interaction.
With this analysis we intend to detect the flaws in the current solutions being presented for interaction 
and evaluate which approaches, if combined, could have a chance of improving 
the guidance methods being used with patients.

With the knowledge gathered from this previous research, a solution will then be implemented and tested with several subjects. 
%After this analysis we will be able to formulate our proposal
%and what it might bring of benefit to this field of rehabilitation.

The document structure is as follows: first, in section \ref{section-related} we will 
explain all the required concepts that are connected to 
this work and overview the state of the art in home rehabilitation systems, giving priority to 
the ones which use some kind of virtual or augmented reality interaction. Also, and most importantly, the feedback strategies used,
not only in these approaches but also in other examples, will be analyzed and compared. 
At the end of this section, an overview will be presented with some conclusions about the current state of the art in multimodal feedback.

In section \ref{section-vision}, our vision will be explained in detail. We fully explain what do we want this solution to be capable of while also giving examples of possible situations in which it would be used.

Further in the end, in section \ref{section-approach}, we will describe our approach with detail, explaining our goals 
while presenting the planned system architecture and evaluation methods to be used.
\subsection{Innovative Contributions}

Studies have already shown that the usage of augmented reality feedback enhances the motor learning of an individual \cite{Sigrist2013}.
Aiming to a home rehabilitation process without the presence of a therapist,
the feedback has the responsibility of guiding the patient and correcting him throughout his tasks.
By experimenting on the several forms of feedback that can be used on a patient, we want to evaluate which 
combinations can be useful to specific tasks and how can they complement each other 
in a way that it creates a clear set of instructions for the patient. 

In this work, we introduce {\it SleeveAR}, a platform that combines multiple sources of awareness feedback to aid and guide during rehabilitation exercises.  
In the end, we want to be able to determine the most appropriate feedback combinations to successfully guide a person through a given movement.
Hence, the aim is to contribute to future augmented reality feedback applications that might need to interact with a user in the clearest possible way.

%The next section presents the state of the art in motor rehabilitation, addressing technological approaches and other important developments.
